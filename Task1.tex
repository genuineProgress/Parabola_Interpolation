\documentclass[12pt]{article}
\usepackage[export]{adjustbox}
\usepackage{geometry}
\usepackage[fleqn]{amsmath}
\usepackage[T2A]{fontenc}
\usepackage[utf8]{inputenc}
\usepackage[russian]{babel}
\usepackage{amsthm}             	\usepackage{graphicx,booktabs,array}
\geometry{letterpaper}
\usepackage{graphicx}
\usepackage{amssymb}
\usepackage{amsmath}
\graphicspath{ {./images/} }
\title{Отчет по практикуму на ЭВМ "Аппроксимация дифференциальной задачи с
помощью метода кусочно-квадратичных конечных элементов"
}
\author{Артем Зданович, 411 группа}
\date{\today}

\begin{document}

\maketitle

\section{Постановка  Задачи}
Аппроксимировать следующую задачу с помощью метода конечных элементов (кусочно квадратичных) и найти решение полученной системы алгебраических уравнений многосеточным
методом при различных h и f
$$-(ku')'+u'+u=f(x),$$ $$u(0)+u'(0)=u(1)=0, k=
\begin{cases}
5/2 & x \leq{0.5} \\
3 & x >0.5
\end{cases} $$
Исследовать построенную разностную схему на устойчивость и сходимость.
\section{Метод решения}
Найдем решение в виде $u(x)=\sum_{i=0}^{N}{c_i\phi_i^h(x)}+\sum_{i=0}^{N-1}{c_i\psi_i^h(x)}$, где $c_i$ -- некоторые коэффициенты, а $\{\phi_i^h(x),\psi_i^h(x)\}$ -- набор базисных функций. Всего их $2N+1$. Явный вид этих функций:

$\begin{cases}\phi_i^h(x)=\frac{(h(i-1)-x)(-h(i+1)+x)}{h^2},i=0,\dots,N \\ \psi_i^h(x)=-4N^2(x-\frac{i}{N})^2+1,i=0,\dots,N-1
\end{cases}$

Пусть наша сетка имеет вид $D_h=
\{x_j=jh,j=0,
\dots,N;Nh=1\}$. Считаем, что точка разрыва функции $k$---$0.5$ принадлежит множеству узлов сетки (т.е. $N$--четно).

Используем метод Галеркина:

$\sum_{j=0}^{N}{c_j^{\phi}(L\phi_j^h(x),\phi_i^h(x)})+\sum_{j=0}^{N-1}{c_j^{\psi}(L\psi_j^h(x),\phi_i^h(x))}=(f,\phi_i^h(x))$ (и аналогичное равенство, где в правой части скалярных произведений стоит $\psi_i^h(x)$).

Кроме того, имеем граничное условие, которое будем использовать позже:

$u(0)+u'(0)=u(1)=0$.

Задача свелась к нахождению $c_i$ из решения СЛУ $Ac=b$. Найдем элементы $A$ (используя пакет Wolfram Mathematica) для каждого вида базисных функций:

$(L\phi_j^h(x),\phi_i^h(x))=\int_{x_i-1}^{x+i+1}(k(\phi_j^h(x))'(\phi_i^h(x))'+(\phi_j^h(x))'\phi_i^h(x)+\phi_j^h(x)\phi_j^h(x)=\frac{16h}{15}+\frac{8k}{3h},i=j$


$(L\psi_j^h(x),\psi_i^h(x))=\int_{x_{i}}^{x_{i+1}}(k(\psi_j^h(x))'(\psi_i^h(x))'+(\psi_j^h(x))'\psi_i^h(x)+\psi_j^h(x)\psi_i^h(x))=\frac{8h}{15}+\frac{16k}{3h},i=j$


$(L\phi_j^h(x),\psi_i^h(x))=\int_{x_{i-1}}^{x_{i+1}}(-k(\phi_j^h(x))'(\psi_i^h(x))'+(\phi_j^h(x))'\psi_i^h(x)+\phi_j^h(x)\psi_i^h(x))=\pm\frac{2}{3}+\frac{7h}{15}+\frac{4k}{3h}, (+$, если  $j=i$; $-$, если $j=i+1)$

%$(L\psi_j^h(x),\phi_i^h(x))=\int_{x_i-1}^{x_{i+1}}(-k(\psi_j^h(x))'(\phi_i^h(x))'+(\psi_j^h(x))'\phi_i^h(x)+\psi_j^h(x)\phi_j^h(x)=\frac{-2}{3}+\frac{7h}{15}+\frac{16k}{3h},i=2l,j=2l+2$

$(L\phi_j^h(x),\phi_i^h(x))=\int_{x_i-1}^{x+i+1}(k(\phi_j^h(x))'(\phi_i^h(x))'+(\phi_j^h(x))'\phi_i^h(x)+\phi_j^h(x)\phi_j^h(x)=\frac{11h}{30}-\frac{2k}{3h},i=j$

Матрица $A$ имеет вид
$$A=\begin{pmatrix} 1 & \frac{4}{h} & \frac{2}{h} & 0 & 0 & \cdots \\
(L\psi_0^h(x),\phi_0^h(x)) & (L\psi_0^h(x),\psi_0^h(x)) & (L\psi_0^h(x),\phi_1^h(x)) & 0 & 0 & \cdots \\
(L\phi_1^h(x),\phi_0^h(x)) & (L\phi_1^h(x),\psi_0^h(x)) &
(L\phi_1^h(x),\phi_1^h(x)) &
(L\phi_1^h(x),\psi_1^h(x)) &
(L\phi_1^h(x),\phi_2^h(x)) &
\cdots \\
0 & 0 & (L\psi_1^h(x),\phi_1^h(x)) & (L\psi_1^h(x),\psi_1^h(x)) & (L\psi_1^h(x),\phi_2^h(x)) & \cdots
\\
\vdots &
\vdots &
\vdots &
\vdots &
\ddots &
\ddots \end{pmatrix}
$$

где первая строка состоит из краевых условий.

Теперь найдем вектор $b$:

$b_i^{\phi}=(f,\phi_i^h(x))=\int_0^1f\phi_i^h(x)dx=\frac{4}{3}hf(hi)$

$b_i^{\psi}=(f,\psi_i^h(x))=\int_0^1f\psi_i^h(x)dx=\frac{2}{3}hf(hi)
$

Следовательно:
$b=(b_0^{\phi},b_0^{\psi},b_1^{\phi},b_1^{\psi},\dots)$
Используем краевые условия:

$u(0)+u'(0)=0\Rightarrow c_1+\frac{4}{h}c_2+\frac{2}{h}c_3=0$

$u(1)=0 \Rightarrow c_^{\phi}_{N}}\phi_{N}^h=0\Rightarrow c_^{\phi}_{N}}=0$.

Остальные коэффициенты вычислим, используя программу на языке C\texttt{++}


\section{Аппроксимация}
Посмотрим на работу алгоритма на примере. Возьмем в качестве правой части $$f(x)=\pi sin(2\pi{x})-\frac{1}{2}(1+4k\pi^2)cos(2\pi{x})+\frac{1}{2})$$
Тогда аналитическим решением задачи будет $sin^2(\pi{x})$.
Аппроксимируем решение с данной функцией $f$ и построим графики обеих функций. Так как наша схемах на многочленах второй степени точна, то она аппроксимирует решение с точностью $O(h^3)$. Имеет место устойчивость и сходимость.
\begin{figure}[h]
\includegraphics[width=12cm]{Analytic_function}
\end{figure}
\section{Результаты}
\clearpage
\begin{table}\sffamily
\begin{tabular}{l*4{C}@{}}
\toprule
\includegraphics[width=12em]{images/Analytic_1.png} &
\includegraphics[width=12em]{images/Analytic_2.png} \\
\includegraphics[width=12em]{images/Analytic_3.png} &
\includegraphics[width=12em]{images/Analytic_4.png} \\
\includegraphics[width=12em]{images/Analytic_5.png} &
\includegraphics[width=12em]{images/Analytic_6.png} \\
\includegraphics[width=12em]{images/Analytic_7.png} &
\includegraphics[width=12em]{images/Analytic_8.png} \\

\end{tabular}
\end{table}

\clearpage
Погрешность - Евклидова норма $u_{analit}$ и $u_{approx}$ умноженная на $h$ при различных $N =
5, 10, 20, 50, 100, 200, 500, 1000$
Зависимость погрешности, от масштаба разбиения представлена ниже:
\begin{figure}[h]
    \centering
    \includegraphics{images/Residual.png}
\end{figure}

\begin{table}[h]
    \centering
    \begin{tabular}{c|c}
         N & Погрешность  \\
        5 &0.42723 \\
10 &0.05336\\
20 &0.01049\\
50 &0.00234\\
100 &0.00108\\
200 &0.00068\\
500 &0.00043\\
1000 &0.00031\\
    \end{tabular}

\end{table}
\end{document}
